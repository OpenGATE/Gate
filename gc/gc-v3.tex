
\documentclass[11pt]{iopart}
\usepackage[pdftex,bookmarks,pagebackref]{hyperref}

\usepackage{epsfig}
\usepackage{color}

\usepackage{amssymb}
\usepackage{amsfonts}
\usepackage{graphicx}
\usepackage{iopams}
\graphicspath{figures/}
\usepackage{lineno}

\newcommand{\dsnote}[1]{{\color{green}[#1]}}
\newcommand{\sjnote}[1]{{\color{red}[#1]}}

%%%%%%%%%%%%%%%%%%%%%%%%%%%%%%%%%%%%%%%%%%%%%%%%%%%%%%%%%%%%%%%%%%%%%%%%%%%%%%%%%%%%%%
\begin{document}

\title[Gate Hadron PET]{Feasibility study for GATE simulation of
  hadrontherapy dose deposition combined with a complete PET imaging
  system.}

%Feasibility of a Monte Carlo full-scaled
%  simulation of patient dose deposition combined with PET imaging
%  during pencil beam spot scanning with $^{12}$C ion beams using GATE
%  on a HPC cluster}

\author{S{\'e}bastien Jan$^{1}$, Thibault Frisson$^{2,3}$, and David Sarrut $^{2,3}$}

\address{$^1$ CEA, Service Hospitalier Fr{\'e}d{\'e}ric Joliot, Orsay, France}
\address{$^2$ University of Lyon, CREATIS-LRMN; CNRS UMR5220; INSA-Lyon, France}
\address{$^3$ University of Lyon, L\'eon B\'erard Cancer Center, F-69373, Lyon, France}
\ead{david.sarrut@creatis.insa-lyon.fr}

%%%%%%%%%%%%%%%%%%%%%%%%%%%%%%%%%%%%%%%%%%%%%%%%%%%%%%%%%%%%%%%%%%%%%%%%%%%%%%%%%%%%%%
\linenumbers
\begin{abstract}

  {\it Objective. We report the feasibility study for performing with
    GATE simulations of the patient dose deposition combined with the
    complete description of a PET imaging device. Dose distribution
    and PET images were obtained from a carbon ion beam irradiation
    using pencil beam spot scanning delivery. This study shows the
    possibility to simulate with the same Monte-Carlo platform such
    complex scanning treatment and imaging processes. We also show
    that it scales well on a large number of CPUs.

    Methods.Using the GATE platform; explicit simulation from the
    beam to the patient and from the patient to the PET scanner
    system. Scaling was assessed on a 1000 CPUs cluster.

    Results.  First full-scale simulation in the context of
    hadrontherapy including dose monitoring with PET imaging. The dose
    monitoring for $^{12}$C irradiation should be realistic and
    feasible for a target dose higher than 1 Gy.

    Conclusions.  GATE is adapted to perform highly realistic
    simulations in the field of hadrontherapy, combining dose and
    imaging system to monitor the dose delivery during a cancer
    treatment. This approach should be use to develop and optimize
    imaging systems for hadrontherapy applications and to determine
    the feasibility of quantitative imaging protocols. }
\sjnote{Pas retravaille l'abstract}
\end{abstract}

%Uncomment for PACS numbers title message
%\pacs{00.00, 20.00, 42.10}
% Keywords required only for MST, PB, PMB, PM, JOA, JOB? 
\vspace{2pc}
\noindent{\it Keywords}: Monte Carlo ; GATE ; Hadrontherapy ; PET monitoring
% Uncomment for Submitted to journal title message
%\submitto{\PMB}
% Comment out if separate title page not required
\maketitle


%%%%%%%%%%%%%%%%%%%%%%%%%%%%%%%%%%%%%%%%%%%%%%%%%%%%%%%%%%%%%%%%%%%%%%%%%%%%%%%%%%%%%%%
\section{Introduction}

Monte Carlo simulations are essential for many medical applications,
especially for imaging and radiotherapy. Numerical approaches for
imaging applications are used to design new devices, to optimize and
study the effect of acquisition parameters on the image quality. They
are also extremely useful to validate and assess compensation methods
and image reconstruction techniques. In the radiation therapy field,
Monte Carlo simulations are used to develop fast dose deposition
algorithms or to characterize beams properties. The
GATE~\cite{Jan2004, Jan2011} open-source simulation platform, based on
the Geant4 toolkit~\cite{Allison2006, Agostinelli2003}, has been
developed and used since 2002 by the OpenGATE collaboration.
The work presented here consider Positron Emission Tomography (PET) as
a method for in situ monitoring of carbon hadron therapy. This method
considers the image of the radioactivity distribution induced by the
nuclear fragmentation processes occurring during the
irradiation~\cite{Enghardt1999}. Such distribution has been shown to
be correlated with the dose distribution~\cite{Moteabbed2011}, leading
to interesting perspectives in dose monitoring, even if exact
quantification is still studied (e.g. the biological washout effects
due to blood perfusion and metabolism, deteriorates this
correlation~\cite{Parodi2008}).

In this scope, we propose to use GATE as a powerful tool to study the
relationship between the PET image quality and the deposited dose. To
our knowledge, GATE is the only Monte-Carlo platform allowing to
simulate in the same framework all the range of physical events
occurring during such a hadron-PET scan. As example, it has been
successfully used in PET imaging~\cite{Buvat2006a} and for
hadrontherapy dose simulations~\cite{Jan2011, Grevillot2011a}. We
defined a realistic simulation setup. This numerical scenario includes
a model of a realistic $^{12}$C pencil beam scanning, a numerical
patient based on a thoracic CT scan and a PET camera model for the
image acquisition system. This study aims to provide a proof of
concept about GATE used to produce hyper-realistic simulations to
estimate the PET efficiency for therapeutic control in the case of
hadrontherapy treatments, to optimize the design of dedicated PET
detectors and to identify the best protocols to control and follow the
deposited dose.

First results of GATE simulations including a carbon ions ($^{12}$C)
irradiation of a lung tumour followed by a full PET acquisition are
presented. These simulations were performed on a supercomputer
(Computing Center for Research and Technology) using around 1000 CPUs
to demonstrate the high scalability of the platform.

%%%%%%%%%%%%%%%%%%%%%%%%%%%%%%%%%%%%%%%%%%%%%%%%%%%%%%%%%%%%%%%%%%%%%%%%%%%%%%%%%%%%%%%
\section{Method}

% \cite{Sommerer2009}

% \cite{Sommerer2006,Pshenichnov2007,Ponisch2004,Parodi2008,
%   Parodi2007,Parodi2002,Parodi2000,Parodi2007a,Hishikawa2002,
%   Fiedler2008,Crespo2006,Enghardt1999,Attanasi2009,Pshenichnov2006}

\subsection{Two steps approach}

We considered the simulation in two parts. The first part simulated
from the $^{12}$C beam at the exit of the nozzle to the
patient. Variables (energy, dose, position) were stored for primary
and secondary particles and specially $\beta^+$ emitters distributions
created during nuclear processes. At this level, we used a phase space
approach and the spatial distributions of $\beta^+$ emitters were
stored. These 3D maps were used as an input files for the PET scan
acquisition which were the second part of the simulation. This two
steps approach was design to give users opportunities to test
different configurations for the imaging system with the same
deposited dose distribution. With this strategy, it is also possible
to model in-beam prompt-gamma dose monitoring
strategy~\cite{Testa2008, Moteabbed2011} if information regarding the
gamma prompts produced by $^{12}$C nuclear reactions during tissue
interactions, are stored in dedicated files. This gamma prompts file
should be also used as a phase space for imaging modeling. This
approach does not induce a penalty regarding the total computing time
of the simulation.

%When created, such emitters were stopped during the simulation and 

\subsection{Simulation setup: target and beam delivery}

\subsubsection{CT Target}

The geometry of the simulation setup is as follows. Target was
imported from a X-Ray Computed Tomograph (CT) image resampled to
isotropic $2^3$ mm$^3$ voxel size (1.7 million voxels). Image was
described with the method proposed in~\cite{Sarrut2008} allowing fast
particle navigation through the matrix of voxels. Hounsfield units
where converted into Geant4 materials thanks to the stoichiometric
calibration method proposed in~\cite{Schneider2000a}. A density
tolerance of $0.1 g/cm^3$ was used leading to $30$ different Geant4
materials, with atomic composition interpolated from 7 initial
materials (Air, Lung, Adipose Tissue, Adrenal Gland, Soft Tissues,
Connective Tissue, Marrow Bone) according to the mass density. The
numerical phantom used for this study is a patient image of a 4D CT
thoracic acquisition including a lung tumour on which an approximated
gross tumour volume (GTV) was manually delineated.

\begin{figure}[!h]
\centering
\includegraphics[width=6cm,height=40mm]{figures/phantomCT.jpg}
\caption{Coronal slice of the thorax CT used for this study. Tumour
  target is clearly visible in the center of the parenchyma on the
  right lung (left of the image).}
\label{fig00}
\end{figure}

\subsubsection{Physics list}

As recommended by the Geant4 Electromagnetic Standard working group
for medical physics applications~\cite{G4EMGroup2010}, we used the
electromagnetic (EM) standard package with the option 3 (Opt3)
parameters-list. Regarding hadronic interactions, we followed the
recommendation of~\cite{Pshenichnov2007, Zacharatou2008}. One
difference was, following~\cite{Polf2009, Peterson2009,
  Grevillot2010}, that we used only the precompound model as inelastic
hadronic model even above 80 MeV (unlike~\cite{Pshenichnov2007}), that
have been found to be closer to measurement in proton
experiments. Very recently, Bohlen et al~\cite{Bohlen2010} proposed an
interesting study comparing nuclear fragmentation models of FLUKA and
Geant4 for carbon ion therapy, and gave indication to better tune
Geant4. Those recommendation will be included in future works.  In the
case of this study, we are focused on the positron emitter production
($^{11}$C and $^{15}$O) for dosimetry monitoring by PET imaging. As it
was demonstrated by Pshenichnov et al.~\cite{Pshenichnov2006} and
illustrates by the table~\ref{tab:CrossSection}, the numerical model
overestimates the cross section production for $^{11}$C and $^{15}$O
of 15\% to 20\%. This will be directly correlated to the PET signal
with an overestimation in the same range.

\begin{table}[htbp]
\begin{center}
  \begin{tabular}{|c|c|c|c|c|c|c|} \hline
    $^{12}$C energy  & \multicolumn{2}{|c|}{212.12 A MeV}  & \multicolumn{2}{|c|}{259.5 A MeV}  & \multicolumn{2}{|c|}{343.46 A MeV}       \\
    \cline{2-7} & Simulation & Experiment & Simulation & Experiment &
    Simulation & Experiment \\ \hline \hline 
    $^{11}$C             & 11.9  &  10.5  &  16.8  &  14.7  &  25.25  &  19.9      \\ \hline
    $^{15}$O             & 2.38  &  2.1  &  3.69  &  3.1   &  6.09   &  5.0        \\ \hline \hline 

\end{tabular}
\end{center} 
\caption{\it Production yields of positron emitter nuclei (per beam particles in \%) in a PMMA phantom: Geant4/GATE simulation versus experimental data.} 
\label{tab:CrossSection}
\end{table}

\dsnote{D'ou tiens tu ces valeurs ? Tes manips ? S'il c'est deja publie dans un papier, pas la peine de le remttre me semble t il.}\\
\sjnote{Valeurs deja publiees par Pshenichnov sur la base de mesure faites au GSI par Parodi mais c est un point fondamental donc je pense que c est bon pour le lecteur d'avoir concretement les valeurs sous les yeux et de ne pas avoir a revenir sur un article precedent. C est quand meme un point clef ou tu vois clairement l'overestimation de la simulation sur la production d emetteurs de positons et ca va donc artificiallement dans le sens d'une amelioration de la qualite d'image PET au final.}

\subsubsection{Scanned beam delivery system}

We considered a virtual beam delivery system inspired from the one
described in~\cite{Kramer2000a}. Scanning is achieved with
superimposition of multiple carbon ion spots varying in lateral and
longitudinal directions, and in energy. The scanning is intended to
spread out the dose such that the target volume receive an uniform and
high dose.  Given a beam direction, the distal plane of the target
volume was sampled with spots positions every 1.5 mm. Each spot was
mono-energetic, have a symmetric Gaussian profile of 5 mm FWHM and was
directed towards its end point in the distal plane. We considered
energy layers every 2 mm (about 3 MeV). We defined a simple pencil
beam model following~\cite{Grevillot2010}. We considered a unique spot
size, even if generally, the size depends on the beam energy. More
advanced descriptions of a pencil beam scanning delivery technique can
be performed~\cite{Grevillot2011} when focus on a real facility. In
Carbon facilities, such as at HIT or GSI, the irradiation delivered by
the synchrotron is pulsed, with spills (beam extraction) during few
seconds (max 2 s) and spill period of about 5s, required for beam
injection and acceleration. We considered here a continuous time
structure and did not split it into spills. We did not also consider
the time required to switch energy between two iso-energy slices,
which is typically in the range of 1-2 s~\cite{Rietzel2010}. The
optimization of each spot intensity was performed according to the
simplified one-dimensional method of~\cite{Kramer2000a} in order to
obtain uniform dose in the target part. Optimization was performed
field by field (single field uniform dose, SFUD~\cite{Lomax1999}). We
used a precomputed database of depth deposited energy profiles in
water. The database contained profiles with energy from 160 MeV/u to
230 MeV/u, with a 0.05 mm depth resolution and was computed with
\sjnote{J ai verifie dans les macro de beam et pour moi l'energie est comprise entre 160 et 230 MeV/u. Peut etre peux tu verifier de ton cote histoire d'etre certain des valeurs ?}
multiple GATE simulations and stored in a database. Water Equivalent
Path Length (WEPL) were computed for all lines going from the virtual
beam source to each spot positions in the distal plane and
intersecting the CT voxels. Total WEPL was estimated by summing the
individual WEPL along all intersected voxels weighted by the
intersection length. WEPL for a given material was computed with an
approximation of the relative stopping power (to water) computed with
the Bethe-Bloch formula: $wepl = \rho \frac{Z/A}{0.555}$, with Z and A
the atomic and mass number of the material.

In the case of this study, we considered 3 fields leading to almost
6000 individual spots. It does not correspond to real treatment plans
but aims at describing representative test case. Three setup of
primary carbon ions flux were launched. Typical irradiation-dose rate
is 5 GyE/min/l~\cite{Noda2007}, corresponding to an intensity of
$1.2\times 10^9$ pps (particles per seconds). Regarding our simplified
pencil beam description, we thus considered that in a range of $10^8$
to $10^9$ particles reach the patient will correspond to 1 Gy to 10 Gy
at the tumour. Table~\ref{tab:setup} shows the beam setup for the 3
configurations.

%dose relat / absolue ; no RBE (?) mentionner LET is available in each voxel
%=> chercher dose max par fraction (pas en GyE)
%cf url send to Seb

%http://totlxl.to.infn.it/mediawiki/index.php?title=Water_equivalent_path_length
% **IMAGE OF ALL BP**
% dsarrut@oronge:~/src/tinytps/BeamLines/RBE1
%% ~/src/tinytps/Simulations/Test2

%~/recherche/grandChallenge/simulations/lung/mac_patient2_3> 
% grep "/gps/source/add" spots1-lung.mac | wc -l 1973
% grep "/gps/source/add" spots2-lung.mac | wc -l 1937
% grep "/gps/source/add" spots3-lung.mac | wc -l 2034

% The two beam descriptions (beam1.mac and beam2.mac) used into
% main.mac, correspond to the physical dose SOBP optimization. The file
% beamRBE.mac corresponds to the results of a SOBP optimization perform
% with a single field on biological dose computed with [Kase2006]
% method.

\subsubsection{Scorers}

Dose was scored with the GateDoseActor described
in~\cite{Jan2011,Sarrut2008} that stores deposit dose and energy
distribution in a 3D matrix of dosels, together with the associated
statistical uncertainty. We choose a matrix of dosels of size $2^3
mm^3$.  We also stored the 3D distribution of points of creation of
$\beta^+$ emitters during the irradiation. We limited emitters scoring
to the two main ones, $^{11}C$ and $^{15}O$~\cite{Pshenichnov2007},
the others ($^{10}C$, $^{13}N$, $^{14}O$, $^{17,18}F$, $^{30}P$, for
those with more than 10 s of half-live) were considering to have
negligible influence on a reconstructed PET image but could also be
stored if needed. The energy deposited by secondary particles and
specially by $\beta^+$ emitters in this case, was taking into account
in the dose and energy 3D matrices.

%\textbf{SEB : est ce que la dose (faible?) d{\'e}pos{\'e}e par ces emitters est prise
%en compte ? A indiquer ici} YES.

%to compare to [attanasi 2009] (events collected during 6 times 10 Gy.)

\subsection{Simulation setup: PET scan for dose monitoring}

The classical approach to model the PET acquisition in the case of dose monitoring for hadrontherapy applications, is to convolve the $\beta^{+}$ emitter map with a Gaussian function which is determined by the point spread function of the PET system. With this simplified method, the scanner sensitivity is not taken into account and this is critical point if we expect to evaluate the relation between the deposited dose and final PET image quality. For this reason, a complete Monte Carlo approach was used to simulate the PET acquisition. This is the second part of the complete simulation setup. It is important to note that it is not an in-beam PET configuration and then there is no prompt gamma contamination during the acquisition. This is the reason which explain that gamma prompt and neutron productions are note taken into account in our physical processes.

\subsubsection{Scanner description}

The modeled scanner for this study is the commercial ECAT EXACT
HR+~\cite{Brix1997} system by Siemens. This scanner is made of four
rings of BGO blocks partially cut into an 8x8 array of crystals
measuring 4.0x4.1x30 mm$^3$ each, resulting in a 82.7 cm diameter
detector cylinder and an axial field-of-view (FOV) length of 15.5
cm. We defined the BGO blocks, a back compartment to simulate the back
scattering induced by the light collection system, the lead
end-shielding, and the patient bed. Figure~\ref{fig:fig0} illustrates
the global geometry of the HR+ scanner. The simulation assumes an
energy resolution for each crystal randomly drawn from a uniform
distribution varying between $20 \%$ and $30 \%$ at 511 keV. The
determination of the hit crystal is based on the crystal with the
highest energy deposition, to which an additional analytical spatial
blurring based on a 2D Gauss kernel can be applied to model the
decrease induced by the photomultiplier tube Anger electronic
system. A global sensitivity factor is also defined for each block to
replicate the sensitivity performance of the scanner; this efficiency
factor is set at $90 \%$ in the case of the HR+. This PET scanner was already modelled and
validated by using GATE and all results are described
in~\cite{Jan2005}.

\begin{figure}[!h]
\centering
\includegraphics[width=40mm,height=40mm]{figures/HR_Simu_Gate_GC.jpg}
\includegraphics[width=40mm,height=40mm]{figures/bloc_detec_HR.jpg}
\caption{Illustration of the HR+ PET scanner simulated with GATE
  (Left). At right, one of the 288 blocks PET detectors which is
  composed by 64 BGO crystals. \dsnote{C'est pas très beau ... (si je
    peux me permettre). Allez, un effort, tu peux faire un truc qui
    pete non ? }\sjnote{Je sais mais il faudrait installer une visu genre VRML pour G4 et faire re-tourner les géometries....ca me fait un peu chi...tu penses que c est indispensable ? Je pourrai aussi virer l'illustration mais je me dis que pour la communauté hadron, c est enfonce le clou sur le fait qu on modelise explicitement le systeme d'imagerie.}}
\label{fig:fig0}
\end{figure}

\subsubsection{PET acquisition, data treatments and image reconstruction}

As for the experimental data, a 350-650 keV energy window and a 12 ns coincidence
time window were applied and the PET acquisition is started just post beam irradiation. The total PET scan acquisition time was 10 minutes. 
To improve the image quantification, the simulated data were
normalized, attenuated corrected and reconstructed using the 3D OSEM
method~\cite{OSEM_ref}. To speed-up the complete processing, we used a
PET analytic simulator, ASIM~\cite{Comtat1999}, with the ECAT HR+
scanner geometry description to produce the normalization
sinogram. For the attenuation correction, coefficient factors (ACFs)
were also calculated with ASIM using the voxelized attenuation map
description provided by the numerical patient phantom.

\subsection{Running GATE on a High Performance Computing machine}

These simulations were running at the French Rechearch and Technology
Computing Center~\cite{CCRT} on the TITANE supercomputer. This is a
cluster integrating 1596 Bull NovaScale R422 servers, with 2 Intel
Xeon 5570 quad core processors each including a memory of 3 Go per
core. With 3192 Intel Xeon 5570 4 cores, TITANE offers a processing
capacity above 90 teraflops, and 25 terabytes of core memory, which
put it in june 2009 in 38th place of the worldwide supercomputer sites
Top500 ranking. The TITANE cluster operates the Bull HPC software
platform that includes the Linux operating system and the global and
parallel Lustre file system. This platform is based on an open source
software integrated and optimized by Bull's HPC competence center in
Echirolles, France. GATE simulations were split into similar jobs
having different random seeds. Because it is a dedicated
supercomputer, we used a static parallelization, each job tracking the
same number of particles~\cite{Camarasu-Pop2010}\sjnote{On n a pas utilise le gatelab dans le cas d'espece - Si tu veux siter ce papier faudrait peut etre tourner la phrase differement non ? Genre - For these specific cluster deployements, some dedicated tools are under developments~\cite{Camarasu-Pop2010}.}.
For imaging part, jobs are splitting following the PET acquisition time and including the positron emitters decay time. 

\subsection{Quantitative estimator}

The table~\ref{tab:setup} summarizes the complete simulation setup from the carbon beam description to the PET scan acquisition. As presented before, 3 beam configurations are defined with 3 dose values delivered at the tumour ($S_{1}$, $S_{2}$, $S_{3}$). To perform our quantitative analysis in this study, the quantitative estimator defined is the $L$ parameter which is the distance between the distal falloff (DFO) and the peak as illustrated on the figure~\ref{fig:dfo}.

\begin{figure}[!h]
  \centering
  \includegraphics[width=65mm,height=45mm]{figures/dfo.jpg}
  \caption{$L$ parameter estimation to define a quantitative estimator in this study.}
  
\label{fig:dfo}
\end{figure}

To study the influence of the full Monte Carlo PET modelling using GATE against a Gaussian smoothing on the $^{11}C$ isotope distribution, we compare the $L$ value provided by each simulation set-up. 
The deviation is estimated by the following expression $\Delta_{L}(S_{1}^{*};S_{1})$. The second estimation aims to quantify the contribution of $^{15}O$ isotope in the PET image in order to estimate if it can be neglected or not in the analysis. For that, protocols $S_{3}^{*}$ and $S_{3}$ are compared and the deviation is expressed by $\Delta_{L}(S_{3}^{*};S_{3})$. 
Finally, to quantify the dose relationship on the image quality, we compared the protocols $S_{3}$, $S_{2})$ and $S_{1})$ with $\Delta_{L}(S_{3};S_{2})$, $\Delta_{L}(S_{2};S_{1})$ and $\Delta_{L}(S_{3};S_{1})$.
 
\begin{table}[htbp]
\begin{center}
\begin{tabular}{|c|c|c|c|c|} \hline
Simulation setup name  & $^{12}$C number  & Tumour dose   & PET simulation  & $\beta^{+}$ emitter  \\
                      &                & (Gy)  & mode & contribution \\\hline \hline

$S_{1}$    & $3.10^{8}$             & 1                     & Full GATE           &           $^{11}$C      \\ \hline
$S_{1}^{*}$    & $3.10^{8}$         & 1                     & Gaussian Smoothing  &      $^{11}$C           \\ \hline
$S_{2}$    & $15.10^{8}$            & 5                     & Full GATE           &       $^{11}$C          \\ \hline
$S_{3}$    & $3.10^{9}$             & 10                    & Full GATE           &           $^{11}$C      \\ \hline
$S_{3}^{*}$ & $3.10^{9}$            & 10                    & Full GATE           &  $^{11}$C + $^{15}$O    \\ \hline \hline 
\end{tabular}
\end{center} 
\caption{\it Setup description: Carbon ions beam, target dose and PET simulation approach.}
\label{tab:setup}
\end{table}

%%%%%%%%%%%%%%%%%%%%%%%%%%%%%%%%%%%%%%%%%%%%%%%%%%%%%%
\section{Results and discussion}

%All figures are presented with the same dynamic in the look up table.
%Averaged statistical uncertainty was close to $XX\%$ for all dosel with more than $50\%$ of the maximum deposited
%dose~\cite{Chetty2006b}.\\ % <---- TO PUT IN RESULTS ???


The figure~\ref{fig:fig1} shows the dose deposited during the
irradiation and the $\beta^+$ emitter maps for $^{11}$C and $^{15}$O
isotopes which are produced by $^{12}$C interactions for the setup $S_{3}$
where 10 Gy were deposited on the tumour. These images are a
qualitative proof of concept of the GATE capabilities to perform
complete and realistic simulations in the field of hadrontherapy
coupled with a nuclear imaging device.

\begin{figure}[!h]
  \centering
  \includegraphics[width=55mm,height=35mm]{figures/3D-Dose_poumon.jpg}\\
  \includegraphics[width=55mm,height=35mm]{figures/3D-PETC11_poumon.jpg}
  \includegraphics[width=55mm,height=35mm]{figures/3D-PETO15_poumon.jpg}
  \caption{Lung dose deposition (Top), and PET images (Bottom) for $^{11}$C isotope (left) and $^{15}$O emitter (right).}
  
\label{fig:fig1}
\end{figure}

The figure~\ref{fig:fig3} shows the comparison between the protocols $S_{1}$ and $S_{1}^{*}$ to illustrate the benefit to model the complete process, from the radiation to the scanner acquisition system. In the case of $S_{1}^{*}$, the PET scanner is modelling by using a smoothing spatial Gaussian function applied to the $\beta^{+}$ emitters distribution maps, such as for example done in~\cite{Parodi2007a} and consequently it does not take into account important physical effects such as material attenuation, photon scattering and the sensitivity of the PET camera including the detection solid angle and the intrinsic detector response. All these effects have an impact on the image quality and finally on the dose quantification with a PET acquisition. It is clearly illustrated on the same figure with image provided by protocol $S_{1}$ where the dose on the tumour is the same (1 Gy) and including a complete PET simulation. The 2 profiles are used to calculate $L$ values and results are summarized in table~\ref{tab:gauss}. The estimation of $\Delta_{L}(S_{1}^{*};S_{1}) = 18.7 \%$ in this configuration, demonstrates that data quantification is biasing  
with a simplified approach to model the imaging system.

\begin{figure}[!h]
  \centering
  \includegraphics[width=6cm,height=40mm]{figures/gaussPET_C11_v2.jpg}
  \includegraphics[width=6cm,height=40mm]{figures/C11_1Gy_v2.jpg}
  \includegraphics[width=65mm,height=45mm]{figures/prof_gaussPET_C11_v1.jpg}
  \includegraphics[width=65mm,height=45mm]{figures/1GyFullPET_v1.jpg}
  \caption{Comparison between protocol $S_{1}^{*}$ and $S_{1}$, for 1 Gy on the target. The profiles are used to quantify the effect of non PET modelling versus a complete Monte Carlo PET simulation.}
  \label{fig:fig3}
\end{figure}

\begin{table}[htbp]
\begin{center}
\begin{tabular}{|c|c|c|} \hline
 Protocol Name                & $L$            & Max peak position       \\
                              &  mm                  & mm                       \\ \hline \hline
$S_{1}$                       & 6.5                  &  119.2            \\ \hline 
$S_{1}^{*}$                   & 8                    &  117.8         \\ \hline 
$\Delta_{L}(S_{1}^{*};S_{1})$         &         \multicolumn{2}{c|}{18.7 $\%$}    \\ \hline\hline
\end{tabular}
\end{center} 
\caption{\it Quantitative estimator to evaluate the effect of complete PET acquisition modelling.}
\label{tab:gauss}
\end{table}

The next point of interest concerns the $^{15}$O isotope contribution in the final PET images an specially regarding the over-lapping effect on the $^{11}$C distribution. As shown in the table~\ref{tab:CrossSection} the $^{11}$C production is higher than $^{15}$O by a factor 4-5 with the used of $^{12}$C beam. The isotope half-life is also an important parameter to estimate the $^{15}$O contribution on the PET image analysis: 2 minutes for $^{15}$O vs 20 minutes for $^{11}$C. If we take into account these 2 effects, with a PET acquisition of 10 minutes (post-irradiation), we can explain what is illustrated by the figure~\ref{fig:fig4} where it is clear that the $^{15}$O production do not affect the quantification. This is also expressed by table~\ref{tab:O15} which summarized comparisons between protocol $S_{3}$ and $S_{3}^{*}$ with a difference estimates by $\Delta_{L}(S_{3}^{*};S_{3}) = 1.1 \%$. Finally, the PET image analysis will not be constrained by the presence of multiple $\beta^+$ emitters and specially by the $^{15}$O. It is not necessary at this level of knowledge and with this set-up parameters, to develop
a dedicated method to correct the $^{11}$C PET image from the $^{15}$O contribution. 


\begin{figure}[!htbp]
  \begin{center}
    \includegraphics[width=6cm,height=40mm]{figures/C11_10Gy_v2.jpg}   
    \includegraphics[width=60mm,height=40mm]{figures/C11_O15_10Gy_v2.jpg}
    \includegraphics[width=75mm,height=50mm]{figures/prof_C11_O15_10Gy_v3.jpg}
    \caption{Comparison between protocol $S_{3}$ and $S_{3}^{*}$ to illustration the $^{15}$O isotope effect on the PET image monitoring. The 2 profiles compare directly a pure $^{11}$C PET image with one including $^{15}$O and $^{11}$C isotopes.}
  \end{center}
  \label{fig:fig4}
\end{figure}

\begin{table}[htbp]
\begin{center}
\begin{tabular}{|c|c|c|} \hline
 Protocol Name                & $L$            & Max peak position       \\
                              &  mm                  & mm                       \\ \hline \hline
$S_{3}$                       & 9.1                  &  116.6                       \\ \hline
$S_{3}^{*}$                   & 9.2                  &  115.3               \\ \hline
$\Delta_{L}(S_{3}^{*};S_{3})$         &        \multicolumn{2}{c|}{1.1 $\%$}     \\ \hline\hline
\end{tabular}
\end{center} 
\caption{\it Quantitative estimator to evaluate the $^{15}$O contribution in the PET image control.}
\label{tab:O15}
\end{table}


\begin{figure}[!h]
  \begin{center}
  \includegraphics[width=6cm,height=40mm]{figures/C11_10Gy_v1.jpg}
  \includegraphics[width=6cm,height=40mm]{figures/C11_5Gy_v1.jpg}
  \includegraphics[width=6cm,height=40mm]{figures/C11_1Gy_v1.jpg}
  \includegraphics[width=75mm,height=55mm]{figures/10-5-1-Gy.jpg}
  \caption{${11}^C$ PET images obtained from the complete imaging simulation for setup $S_{3}$
    (10 Gy), $S_{2}$ (5 Gy) and $S_{1}$ (1Gy). Histogram distribution for quantitative analysis on PET
      images versus dose deposited at the tumour level. }
  \end{center}
  \label{fig:fig5}
\end{figure}



\begin{table}[htbp]
\begin{center}
\begin{tabular}{|c|c|c|} \hline
 Protocol Name                & $L$            & Max peak position       \\
                              &  mm                  & mm                       \\ \hline \hline
$S_{3}$                       & 9.1                  &  116.6                       \\ \hline
$S_{2}$                       & 7.8                  &  116.6                 \\ \hline
$S_{1}$                       & 6.5                  &  119.2            \\ \hline 
$\Delta_{L}(S_{3};S_{2})$         &       \multicolumn{2}{c|}{16.5 $\%$}  \\ \hline
$\Delta_{L}(S_{2};S_{1})$         &        \multicolumn{2}{c|}{20.0 $\%$} \\ \hline
$\Delta_{L}(S_{3};S_{1})$         &        \multicolumn{2}{c|}{40.0 $\%$}  \\ \hline \hline
\end{tabular}
\end{center} 
\caption{\it Quantitative estimators to evaluate the relation between the tumour deposited dose and the $^{11}$C PET image monitoring.}
\label{tab:dose}
\end{table}
\sjnote{Je dois completer a ce niveau dans l'explication}
These results illustrate clearly the high correlation and sensitivity between the dose on the target and the capability to have a quantitative analysis with the control PET imaging.

%%%%%%%%%%%%%%%%%%%%%%%%%%%%%%%%%%%%%%%%%%%%%%%%%%%%%%%%%%%%%%%%%%%%%%%%%%%%%%%%%%%%%%%
\clearpage
\section{Conclusion}

\sjnote{Pas retravaille la conclusion}

This study demonstrates the capability of GATE to perform realistic
simulations in the fields of hadrontherapy and nuclear imaging, for
in-vivo dose monitoring. Thanks to the high scalability of the code,
this work also shows that the GATE platform is clearly an interesting
tool to produce large Monte-Carlo data using a computing center. This
characteristic should be an advantage for the production of database
to work on protocol optimization, or to define and validate analytical
models for dose monitoring. We demonstrated in this study that to
perform a PET imaging monitoring, the deposited dose should be higher
than 1 Gy for a target volume close to 4 ml. According to the relative
contribution of the different $\beta^+$ emitters for a
post-irradiation PET acquisition longer that 10 minutes, we conclude
that it is not necessary to develop some isotope deconvolution methods
to quantified the correlation between $^{11}$C distribution and dose
deposited by $^{12}$C.

We also illustrated the importance of using a realistic PET scanner
modeling versus using a Gaussian function response as it is generally
done. We showed that GATE is adapted to design imaging systems for
hadrontherapy, to optimize the sensitivity and the quantitative
performances for monitoring the beam delivery. It is well known that a
major challenge concerns the detection with a very low statistic of
events where the performances of the imaging system and the image
reconstruction algorithm are crucial. Some approaches are focused on
the 4D algorithm developments~\cite{Fall2011} and, in our opinion, it
will be essential to validate all these new approaches to produce very
realistic data sets of irradiation protocols associated to the imaging
system for therapeutic control.

%NEXT BIG POINT : TIME STRUCTURE OF THE BEAM + phase-space


%%%%%%%%%%%%%%%%%%%%%%%%%%%%%%%%%%%%%%%%%%%%%%%%%%%%%%%%%%%%%%%%%%%%%%%%%%%%%%%%%%%%%%%
%\bibliographystyle{jphysicsB}
%\bibliographystyle{agsm}
\bibliographystyle{unsrt}
%\bibliographystyle{jmr}
%\bibliographystyle{agsm}

%\bibliographystyle{unsrt}
\bibliography{gc}
%%%%%%%%%%%%%%%%%%%%%%%%%%%%%%%%%%%%%%%%%%%%%%%%%%%%%%%%%%%%%%%%%%%%%%%%%%%%%%%%%%%%%%%
\end{document}
